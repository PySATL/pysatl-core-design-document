\chapter{Введение}
\section{Назначение системы}
Вычислительное ядро проекта PySATL предназначено для представления и обработки вероятностных распределений в программной форме. Ядро предоставляет средства для задания распределений и их семейств, выполнения операций над ними и построения более сложных структур путём функциональных и алгебраических преобразований.

Система поддерживает задание как конкретных распределений с определёнными параметрами, так и абстрактных семейств, из которых могут быть получены конкретные экземпляры. Кроме того, предусмотрена возможность определения пользовательских распределений и преобразований, расширяющих базовые возможности.

Ядро служит универсальной основой для статистических и вероятностных вычислений в рамках проекта PySATL и может использоваться другими подсистемами при построении моделей.

\section{Область применимости}
Вычислительное ядро \texttt{core} используется во всех подсистемах проекта PySATL, где требуется работа с распределениями вероятностей. Оно предназначено как для непосредственного вычисления характеристик распределений (например, плотности, функции распределения, квантилей), так и для построения и трансформации более сложных моделей на их основе.  Оно может быть использовано:
\begin{itemize}[compact]
    \item при определении конкретных распределений, используемых в анализе данных;
    \item для задания пользовательских распределений, комбинации распределений и создания новых семейств;
    \item при трансформации распределений через функциональные отображения;
    \item в задачах символьной или численной обработки распределений.
\end{itemize}

Вне проекта PySATL ядро может быть применимо в любых системах, где необходима гибкая и расширяемая работа с вероятностными распределениями, особенно в контексте численных симуляций, статистического моделирования и прикладного машинного обучения.

\section{Общие сведения о системе}
Ядро представляет собой модульную систему, реализованную на языке Python, предназначенную для работы с вероятностными распределениями, их семействами и преобразованиями. Архитектура системы построена на разделении функциональности между независимыми компонентами, каждый из которых отвечает за определённый класс задач.

Компоненты взаимодействуют друг с другом через чётко определённые интерфейсы. Такая организация позволяет изолировать ответственность отдельных частей системы, обеспечивать гибкость при расширении функциональности и облегчать сопровождение кода.

Некоторые интерфейсы, предоставляемые одним модулем, используются другими модулями для построения более сложных вычислений или абстракций. Это позволяет комбинировать базовые элементы в составные структуры и формировать цепочки преобразований.

Проект спроектирован таким образом, чтобы допускать расширение без модификации существующих компонентов, в соответствии с принципами модульности и открытости/закрытости. Это обеспечивает стабильную основу для развития ядра и его интеграции с другими частями проекта PySATL.
